\chapter{IN-MEMORY INTEGRATION OF EXISTING SOFTWARE COMPONENTS FOR PARALLEL
ADAPTIVE UNSTRUCTURED MESH WORKFLOWS}
\label{chp:inmem}
\blfootnote{Portions of this chapter have been submitted to:
            \bibentry{smith2017building}}
\subimport{imp/}{impIntro}
This chapter presents the use of in-memory component coupling techniques that
avoid filesystem use for three different unstructured mesh-based, parallel,
adaptive workflows.
These demonstrations highlight the need for in-memory coupling techniques that
are compatible with the design and execution of the analysis software involved.
Key to this compatibility is supporting two interaction modes: bulk and
atomic information transfers.

\section{PHASTA}\label{sec:imp_phasta}
\subimport{imp/}{impPhasta}
\subimport{imp/}{impPhastaDambreak}
\subimport{imp/}{impPhastaPerformance}
\section{Albany Solderball}\label{sec:imp_albany}
\subimport{imp/}{impAlbany}
\subimport{imp/}{impAlbanySolderball}
\section{Omega3P Cavity Frequency Analysis}\label{sec:imp_omega3p}
\subimport{imp/}{impOmega3p}

\section{Summary}
In-memory parallel adaptive workflows for three applications have been
demonstrated using bulk data streams, bulk APIs, and a combination of bulk and
atomic APIs.
The in-memory transfer of data was significantly faster than
file-based transfers for PHASTA and Albany, while the memory overhead for Omega3P
was insignificant.
Key to the three couplings was the use of PUMI's component APIs for queries to, and
modifications of, the mesh, its partitioning, and its associated fields.
