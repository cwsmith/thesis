\chapter{CONCLUSIONS AND FUTURE WORK}
\label{chp:conclusion}
\section{Conclusions}

As we move towards the exascale computers being considered \cite{Cross10,Exa10},
it is clear that one of the few effective means to construct parallel adaptive
simulations is by using in-memory interfaces that avoid filesystem interactions.
Of course, the cost of refactoring existing large-scale parallel partial
differential equation solvers to fully interact with the type of structures and
methods used by mesh adaptation components is an extremely expensive and
time-consuming process.
To address these costs, we presented approaches for in-memory integration of
existing solver components with mesh adaptation components, discussed how code
changes can be minimized, and demonstrated the performance advantage within
adaptive simulations.
Demonstrations with the massively parallel computational fluid dynamics, solid
mechanics, and electromagnetics adaptive workflows showed orders of magnitude
performance improvements in I/O procedures using in-memory coupling instead of
files at up to 16Ki processes of the ALCF Theta system.
In addition to efforts on developing in-memory approaches with the PHASTA,
Albany, and Omega3P solvers, efforts are underway to interface other
state-of-the-art solvers including NASA's FUN3D~\cite{anderson1999achieving} and
LLNL's MFEM~\cite{mfem-library}.

Parallel scalability of these workflow components is maintained with
new methods to dynamically balance the computational domain.
These methods work directly on the unstructured mesh alongside traditional graph
and geometric methods to quickly reduce the source of imbalance the
consuming workflow component is sensitive to.
This approach improves the linear algebra work performance of PHASTA
computational fluid dynamics by 28\% over a graph-based partition, and improves
scaling from 0.82 to 1.14, on 512Ki processes of the ALCF Mira system.

As the high performance computing community prepares for near-exascale systems
at the national laboratories, and near-petascale systems become the norm for academia and
advanced industry users, we continue to advance parallel unstructured mesh-based
simulation technologies for efficiently exploiting the massive amount of
computing power on hand and on the horizon.
The challenge is clear; adapt to hardware changes and application needs by
applying new programming and algorithmic approaches while maintaining support
for the existing user base.
We address this challenge by focusing a significant portion of our efforts on
leveraging the wealth of existing components and the thousands of person-hours
invested in them.
In the era of distributed memory message passing between many-core nodes, this
has been a broadly successful approach and is expected to continue well into the
next decade.

\section{Future Work}

\subsection{Partitioning and Load Balancing}

The ParMA load balancing algorithms that worked
directly on unstructured meshes are being generalized to support applications
that use a different mesh distribution (e.g., node partitions), or simply have
information and dependencies between them that can be represented with a graph.
EnGPar~\cite{engpar_github} will implement the ParMA diffusive
algorithms and multi-level graph partitioning procedures using data-parallel
operations on a graph structure using multiple edge-types to represent different
application information dependencies.
Efforts are underway to implement the diffusive algorithms and explore the use
of the Kokkos~\cite{edwards2013kokkos} programming model for
performance portability.

\subsection{In-memory Component Coupling}

Parallel system vendors have started to deploy an additional layer
in the memory hierarchy between the parallel filesystem and main memory (DRAM)
on each node.
The additional layer is implemented by Cray DataWarp devices on the Cori
XC40 system at the National Energy Research Scientific Computing Center.
In the 2018-2019 Aurora system at the Argonne Leadership Computing Facility the
layer will be implemented with Intel SSDs.
By design, we expect transfer tests operating on the new layer to be slower
than data streams and APIs operating out of DRAM or high bandwidth memory.
However, depending on the locality of the SSD or DataWarp devices to compute
nodes, the performance could vary significantly and should be evaluated.

\subsection{Unstructured Mesh-Based Workflows}

Application support using the PHASTA-PUMI and Albany-PUMI
parallel in-memory workflows continues with a fluid structure interaction
problem and an additive manufacturing analysis.
In both cases the coupling approaches developed are reused while specific
aspects of the finite element and mesh adaptivity components are modified.
For PHASTA, developments are focused on implementing the discontinuous
Galerkin method to account for interactions across the solid-gas interface.
In the Albany additive manufacturing workflow the focus is on mesh
adaptation methods to support the evolution of the geometry as layers of
material are deposited.

Ongoing interactions with industrial users are focused on two applications with
evolving geometry.
The first application is simulating flow in a gas filled chamber with a moving
displacer.
Implementation of this workflow requires combining mesh motion to track the
moving geometry and mesh adaptation to maintain element quality.
The second application uses MeshAdapt to support simulation of an underground
reservoir with evolving geometry.
