\noindent Title: Improving Scalability of Parallel Unstructured Mesh-based Adaptive Workflows\\
Author: Cameron W. Smith\\
Advisor: Mark S. Shephard\\
Date: Thursday April 6th, 2017 at 2pm\\
Location: CII 4003

Parallel adaptive unstructured mesh based simulations are most effectively
implemented using multiple existing software components.
At scale, performance bottlenecks can exist within a component or during the
transformation and transfer of data between them.
The thesis addresses these limitations with massively parallel load balancing
and in-memory inter-component data transfer methods.

Given an existing mesh distribution across the processes running on a parallel system,
a partition, dynamic load balancing methods determine which mesh elements should
be moved between processes to reduce the imbalance and communication costs.
For part counts numbering in the tens of thousands, multi-level methods
operating on the dual graph of the mesh are sufficient, but beyond this
concurrency level these methods often fail due to memory usage that increases
significantly with process count.
The goal of the Partitioning using Mesh Adjacencies, ParMA, software is to
perform efficient multi criteria dynamic load balancing of unstructured meshes
by directly using the existing mesh adjacency information.
Results will demonstrate the ability of ParMA to dynamically re-balance meshes
for multiple criteria with billions of mesh regions on over one million
processors.

In massively parallel simulations requiring frequent adaptation, the reading and
writing of files between component executions introduces significant
computational overheads largely due to the latency of writing and reading to disk.
Thus, interfacing simulation components without files for adaptation greatly
increases the performance of the application.
An in-memory coupling approach using data streams that mimic the use of files
and limit code modifications will be discussed and demonstrated for a monolithic
mixed C/C++ and FORTRAN based computational fluid dynamics (CFD) analysis code,
a C++ hp-adaptive finite element framework for linear accelerator frequency
analysis, and a C++ multi-physics framework.
Scaling results to 16Ki processes are provided for the coupling of the CFD
analysis code with a mesh adaptation component.
