\chapter{IMPROVING UNSTRUCTURED MESH PARTITIONS FOR MULTIPLE CRITERIA USING MESH
ADJACENCIES}
\label{chp:ptnImprovement}
\blfootnote{Portions of this chapter have been submitted to:
C.~W. Smith, M.~Rasquin, D.~Ibanez, K.~E. Jansen, and M.~S. Shephard,
  ``Improving unstructured mesh partitions for multiple criteria using mesh
  adjacencies,'' submitted for publication.
}

\section{Introduction}
Parallel simulation-based engineering workflows using unstructured meshes
require adaptive methods to ensure reliability and
efficiency~\cite{shephard2013bringing}.
Starting with a problem specification on a geometric model~\cite{OBaBea,SheBea},
an effective workflow automatically executes parallel mesh
generation~\cite{tendulkar2011parallel}, analysis, and analysis-based
mesh~\cite{chitale2014anisotropic,Sahn06} and/or
model~\cite{ramm2003error} adaptation.
The analyze-adapt cycle is repeated until a desired level of solution accuracy
is reached.
Between each step in the cycle is an opportunity to improve scalability and
efficiency through dynamic partitioning.

Current dynamic load balancing methods do not effectively reduce imbalances to
the levels needed by applications capable of strong scaling to the full size of
leadership class petascale systems.
This chapter presents a scalable approach that quickly reaches the required
imbalance levels for multiple criteria by pairing Partitioning using Mesh
Adjacencies (ParMA) with current partitioning methods.

\subimport{parmaimprovement/}{parmaSummaryThesis}

\section{Unstructured Mesh Partitioning} \label{sec:partitioning}
\subimport{parmaimprovement/}{parmaBackground}

\section{Dynamic Partitioning Requirements} \label{sec:requirements}
\subimport{parmaimprovement/}{parmaRequirements}

\section{Partitioned Mesh Representation} \label{sec:pumi}
\subimport{parmaimprovement/}{parmaPtnmesh}

\section{Partition Improvement} \label{sec:parma}
\subimport{parmaimprovement/}{parmaPtnimp}

\section{Targeting} \label{sec:targeting}
\subimport{parmaimprovement/}{parmaTargets}

\section{Entity Selection} \label{sec:entSelection}
\subimport{parmaimprovement/}{parmaSelection}

\section{Results} \label{sec:results}
\subimport{parmaimprovement/}{parmaResults}

\section{Summary}
ParMA enables scalable data transformations within components by providing fast,
multi-criteria, dynamic load balancing procedures that execute efficiently on
over one million cores of leadership-class parallel systems.
These procedures rely on part-level (graph distance) and
entity-level (cavity size, surface area) heuristics to define the traversal
order over the part boundary, which elements to select for migration, and the
destination part that should receive the elements.
In addition to the selection heuristics are mechanisms to reject migration
requests if they harm the imbalance of higher priority mesh entities
(cancellation) and to gracefully stop ParMA if beneficial changes can no longer
be made (stagnation avoidance).
The net result of these efforts is a load balancing method that outperforms the
previous versions of the approach (LIIPBMod) in quality and run time, and,
versus a leading graph-based method, improves performance of a massively
parallel CFD code at 512Ki processors by 28\%.
