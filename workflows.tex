\chapter{CONSTRUCTING PARALLEL ADAPTIVE WORKFLOWS}
\blfootnote{Portions of this chapter previously appeared in:
            \bibentry{shephard2013bringing}}
\blfootnote{Portions of this chapter previously appeared in:
            \bibentry{smithXsede15}}
\label{chp:workflow}
\subimport{workflows/}{wrkflwIntro}
\subimport{workflows/}{wrkflwImpediments}
\subimport{imp/}{impCompInteractions}
\subimport{workflows/}{wrkflwComponents}

\section{Summary}
Parallel workflows for unstructured mesh-based, adaptive simulations are most
effectively constructed from existing software components.
Ideally, these components provide APIs that support bulk (sets of data
over a significant portion of the domain on a given process) or atomic (sets of
data over small sub-domains on a given process) queries and modification
procedures to encapsulated state information.
When the existing software component does not provide these APIs, or provides a
very limited set of them, but supports POSIX file-based interactions, then the
bulk transfer of data into and out of the component is well-supported by
data streams.
Components supporting these bulk and atomic transfers are easily coupled with
other unstructured mesh-based components when their operations are based on a
mesh classified against a geometric and partition model.
PUMI uses this approach for its APIs that query and modify the mesh, its
distribution across processes, and its relation to the geometric model.
PUMI's unstructured mesh components include~\cite{ibanez2016pumi}:
\begin{itemize}
\item PCU - neighborhood-based non-blocking collective communication routines
\item GMI - geometric modeling queries supporting discrete models
  and, optionally, Parasolid, ACIS, and Simmetrix GeomSim models using
  the Simmetrix SimModSuite library
\item MDS - array-based modifiable mesh data structure
\item APF\_MDS - partitioned mesh representation using MDS
\item Field - describes the variation of tensor quantities over the domain
\item ParMA - multi-criteria dynamic load balancing
\item MeshAdapt - parallel, size field driven local refinement and coarsening.
\end{itemize}
