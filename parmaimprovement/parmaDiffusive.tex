Diffusive partitioning methods efficiently improve an existing partition
by transferring load between neighboring parts.
Load transfer can be coordinated globally or locally.
Global load transfer selects elements to minimize either the total
weight of transferred elements, or the maximum weight transferred in to or out
from a part
\cite{hu1999improved,Hu1998,meyerhenke2009graph,ou1994parallel,schloegel1997multilevel,schloegel2001wavefront,walshaw1995dynamic}.
Alternatively, local load transfer iteratively moves elements from heavily loaded to
less loaded parts
~\cite{subramanian1994analysis,cybenko1989dynamic,schloegel1997multilevel,willebeek1993strategies}.
This approach can have significantly lower overall computational costs if
the total amount of transferred load is controlled.
Control is typically exerted through greedy heuristics.
These heuristics first determine the amount of load to transfer between
neighboring parts, and then select elements to satisfy the transfer requirement.
Fiduccia~\cite{Fiduccia1982} and Kernighan~\cite{Kernighan1970} proposed
selecting elements based on the subsequent part quality improvement.
In Zhou's work these methods are shown to be highly scalable given a
distributed mesh representation~\cite{Zhou2010,zhou2012unstructured}.
