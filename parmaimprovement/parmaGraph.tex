Graph-based partitioning methods define an assignment of weighted graph
nodes to $k$ parts such that each part has the same total weight and the
inter-part communication costs are minimized.
A graph is constructed from an unstructured mesh by selecting one dimension of mesh
entity (i.e., vertices, edges, faces, or regions) to define graph nodes, and one mesh
adjacency between the selected entity dimension to define graph edges.
At a higher level, the goal of this selection is to represent a work unit with
the graph node and an information dependency between two work units by a graph
edge.
3D element-based finite element and finite volume codes typically select mesh
regions for graph nodes and mesh faces shared by elements for graph edges.
This selection results in the unique assignment of mesh regions to parts,
which enables efficient local execution of element-level
computations~\cite{hughes2012finite}.

Parallel, multi-level, graph-based partitioning methods produce
high quality partitions with tens of thousands of parts in a fraction of the
time needed by most analysis
procedures~\cite{catalyurek2013umpa,karypis1999parallel, lasalle2013multi,
schloegel2002parallel}.
One approach to generalize these methods to represent more complex information
dependencies uses hypergraphs.
A hypergraph is defined as a set of weighted nodes and hyperedges.
Hyperedges differ from graph edges in that they represent dependencies
between multiple graph nodes and, in doing so, have the ability to better
model the communication costs of an
application~\cite{catalyurek1999hypergraph,catalyurek2009repartitioning}.
As with graph-based partitioning, the goal of hypergraph partitioning is to
balance the node weight across the $k$ parts while minimizing a
hyperedge-based objective function.
Boman and Devine propose constructing the hypergraph from an unstructured mesh
by creating one hypergraph node for each mesh region (in 3D), as is done in the
graph-based construction, and a hyperedge connecting the mesh regions
bounded by each mesh vertex.  %this technique was not published
This richer representation results in algorithms that are more compute and memory
intensive relative to graph-based methods.
