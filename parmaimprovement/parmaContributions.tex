Zhou's 2010 work~\cite{Zhou2010} defines the LIIPBMod algorithm for reducing
vertex imbalance and the number of vertices on part boundaries while indirectly
trying to limit the increase of element imbalance.
In 2012, Zhou~\cite{zhou2012unstructured} executes a strong scaling study of a
massively parallel computational fluid dynamics application using partitions
created with (hyper)graph partitioners and LIIPBMod.
Our work, ParMA, defines new algorithms for balancing all entity dimensions
(vertices, edges, faces and regions), with weights, while reducing the number of
vertices on the part boundaries, the number of disconnected components, and the
average number of neighboring parts.
ParMA developments were guided by Zhou's work for vertex balancing.

Our work, relative to Zhou's, demonstrates multi-entity balancing on up to 3.5
times more parts, 1Mi, with up to two times smaller parts, 1100 elements.
Like Zhou, we focus on balancing tetrahedral meshes.
We also support balancing mixed and other monotopological meshes (e.g., all
quadrilaterals or all hexahedra).


ParMA's implementation relies on the PUMI parallel unstructured mesh
infrastructure, and inter-process communication algorithms detailed by Ibanez et
al.~\cite{ibanez2016pumi}.
We refer readers to this paper for details on the element migration procedure
and neighborhood communications for information exchange.
