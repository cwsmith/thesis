ParMA reduces the peak imbalance of multiple entity dimensions by
iteratively migrating some mesh elements from heavily loaded parts
to neighboring parts with less load.
The entity dimensions to balance are defined by an application specified priority
list.
For example, if element$>$vertex is specified then the algorithm prioritizes
improvements to element balance over vertex balance.
The greater-than relation indicates that element balance improvements are
allowed to degrade the vertex balance, but vertex balance improvements cannot
degrade the element balance.
The balance of unlisted entity dimensions (edges and faces in this example) are not
considered and may be degraded.
If vertex$=$element is specified, then the algorithm considers the balance of mesh
elements and vertices equally important.
When equal priority is specified, the lower-dimension entities are processed
first as improvements to their balance tends to improve the
balance of the entities they bound (higher-dimension entities).
The target imbalance for each listed entity dimension is specified by the
application as $tgtImb^d$ where $d \leq d_{max}$ (the maximum dimension entity in a
mesh).
Applications which perform work on entities regardless of their ownership define
the imbalance of a part, $I^d_p$, as the weight of mesh entities of dimension $d$
existing on part $p$ divided by the average weight of dimension $d$ entities per
part.
The weight of a mesh entity is set to one when it is not specified by the
application.
The maximum imbalance of dimension $d$ entities across all parts is noted as
$I^d$.

ParMA iterative diffusion is executed for each specified entity dimension by
descending priority.
Iterations stop if the target imbalance ($tgtImb^d$) is reached, if no migration
opportunities remain (discussed in Section~\ref{sec:stagnation}), or if a
maximum number of iterations is reached.
Each diffusive iteration has four steps~\cite{subramanian1994analysis}.
First, neighboring parts exchange local information (e.g., the weight of
mesh entities) using PCU neighborhood communication
procedures~\cite{ibanez2016hybrid}.
Next, each part determines how much load needs to be migrated and
where it needs to go, targetting, and marks elements for migration, selection.
The final step, migration, moves the marked elements to their defined
destinations using the collective PUMI migration procedure~\cite{ibanez2016pumi}.

The targeting and entity selection steps are detailed in the following sections.
