Geometric methods represent information via spatial coordinates, and
relations via distance; the closer two pieces of information
are the stronger their relation.
The exclusive use of coordinate information significantly reduces the memory
requirements of these methods relative to (hyper)graph methods that rely on
topological relations~\cite{harlacherMortonSFCvsParmetis2012}.
Along with the lower memory cost, the spatial sorting procedures used by
geometric methods are also computationally cheaper than the topological traversals
needed by graph methods.
The lower computational and memory usage costs come at the expense of
significant increases in inter-part
communications~\cite{schambergerSFCpartition2005}.
For applications that require frequent balancing through, the resulting
communication overheads may be offset by the time saved computing the partition
~\cite{harlacherMortonSFCvsParmetis2012}.

Geometric recursive sectioning methods can quickly compute well-balanced
partitions for a single entity dimension
~\cite{zoltan2,devineMultiJagged2015,SauleJagged2012,TaylorRIB,williamsRIB}.
Recursive coordinate~\cite{bergerRib1987} (RCB) and inertial
~\cite{simon1991partitioning,TaylorRIB,williamsRIB}
bisection (RIB) methods
recursively cut the parent domain; RCB along a coordinate axis and RIB
perpendicular to the parent domain's principal direction.
Multi-sectioning techniques~\cite{devineMultiJagged2015,SauleJagged2012} can be
considered extensions of the recursive coordinate bisection methods as they
define cuts along coordinate axis, but do so with multiple parallel cut planes
at each recursion.

Partitioning methods using space-filling curves (SFC) produce partitions
of similar quality to RCB and RIB.
For 3D unstructured meshes Hilbert~\cite{skilling2004programming} and Morton
curves have been used effectively by the Zoltan~\cite{devine2005new}
and SPartA~\cite{harlacherMortonSFCvsParmetis2012} packages,
respectively.
Given the simplicity of SFC partitioning methods (encoding, sorting, then
splitting) a high degree of on-node and inter-node concurrency is possible.
For example, a constant time Hilbert curve encoding procedure (spatial
coordinates to curve position)~\cite{skilling2004programming} and its subsequent
sorting has been demonstrated on shared-memory devices using a data-parallel
implementation~\cite{ibanezthesis} and a two-collective splitting approach is
used by SParTA.
As an added benefit, sorting provides a cache efficient layout
of the mesh entities for subsequent mesh-based operations that benefit from
topological locality~\cite{gpsReordering1976,zhou2010adjacency}.
