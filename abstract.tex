\specialhead{ABSTRACT}
High performance parallel adaptive simulations operating on leadership class
systems are constructed from multiple pieces of software developed over many
years.
As increasingly complex systems are deployed new methods must be created to
extract performance and scalability.
This thesis addresses two key scalability limitations for unstructured
mesh-based simulations.

Attaining simulation performance at ever higher concurrency levels requires
increasing the performance of transformations within each procedure, as well as the
transfer of data between procedures.

Controlling the transformations requires distributing the work evenly across the
processors while executing efficient data transfers requires local operations
that avoid shared or contended resources.
This thesis addresses these requirements through multi-criteria load balancing
procedures and in-memory data transfer techniques.

Partition improvement methods defined in this work enable improved application
strong scaling on over one million processors through careful control of the
balancing requirements.
Applied to a computational fluid dynamics simulation running on 524,288
processes with 1.2 billion elements these methods reduce the time of the
dominant computational step by up to 28\% versus the best existing methods.

The scalable data transfer requirement is addressed through an
in-memory functional coupling that avoids the high cost of fileystem access.
The methods developed are applied to adaptive simulations in which
the time required for information exchange is reduced by over an
order of magnitude versus file-based couplings.
Three additional simulations for industrial applications are then provided that
highlight an in-memory coupling and the automation of key simulation processes.
