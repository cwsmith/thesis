\section{Approach to Eliminating Technical Impediments}\label{sec:impediments}

The execution of parallel simulations typically requires a workflow that couples
multiple simulation procedures.
For example, a single physics simulation requires linking geometric modeling
systems with mesh generators and a mesh-based solver.
For multiple reasons, ranging from a single package's best practices and
validation processes, to the interactions of multiple organizations or companies
doing different steps in the process, these capabilities are typically not provided
within a single software system and thus the effective coupling of multiple
software packages is critical.
The need to couple packages becomes more acute in multi-physics and multi-scale
problems where procedures based on different models using different
physical parameters must be coupled across
scales~\cite{amsiDelalondre2010,FrantzDale2010}.
The ability to support these workflows is made even more complex when specific
computationally intensive simulation steps must be executed on a massively
parallel computer.
One approach to couple software packages for the effective construction
of simulation workflows is through the use of functional interfaces.

The most effective way to design simulation workflows are based on geometric
models of the computational domain.
Accordingly, CAE analysis tools have begun to move from mesh-based problem
definitions to more geometry-based definitions.
These simulations are typically executed using a graphical user interface that
can accept CAD information.
Many of these interfaces are oriented toward a specific vendor's set of CAE
analysis tools while some others are oriented to interact through APIs
supporting general interactions with geometry~\cite{BeaWal} and simulation
attribute information.
In those projects where the workflow developed employed software from several
sources, we found using a generalized interface that can interact with geometry,
attributes, and analysis procedures to be quite
effective~\cite{ibanez2016pumi,tautges2001cgm,haimesEngSketchPad2013}.
Given the problem definition, most analysis procedures require the domain be
decomposed into a mesh of simple shapes.
The greatest flexibility is provided when the parallel analysis procedures can
accept an unstructured mesh; a decomposition of the domain into elements of
various topological type (e.g., tetrahedrons, hexahedrons, prisms, and pyramids
in 3D), order (i.e., straight sided or curved), quality, and volume.
Mesh-based finite element~\cite{hughes2012finite} and
finite volume~\cite{abgrallFiniteVolume2016} analysis methods are available for
many classes of problems.
The use of these methods allows both the application of fully automatic mesh
generation and adaptive mesh control.

Workflow development for various academic and industrial applications has
provided valuable lessons on the design of interoperable interfaces.
One key lesson is that there are some common high-level interfaces that can be
defined for coupling many of the simulation procedures.
These interfaces are primarily focused on methods to load inputs into the
data structures of analysis procedures, and subsequently extract
simulation data from those structures.
For procedures using a geometry-based problem definition and a solution field
interface, this approach allows the fast integration of multiple
meshing, analysis and visualization procedures, as well as supporting the
ability to quickly replace any of their implementations.
The typical initial implementation of these interface methods tend to pass
information between major procedures using files.
Although the most straightforward implementation, file I/O (serial or parallel)
is a major bottleneck in the execution of large-scale parallel simulations.
When the method used to execute the coupling of procedures is through APIs, it
is conceptually straightforward to bypass the use of files and transfer
information directly between data structures.
However, the effective implementation, given existing software packages,
is reasonably complex.
