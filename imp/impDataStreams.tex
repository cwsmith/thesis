\subsection{Data Streams} \label{sec:imp_DataStreams}
Components can pass information and avoid expensive filesystem operations
through the use of buffer-based data streams.
This approach is best suited for components already using
POSIX C \texttt{stdio.h}~\cite{posixstandard}
or C++ \texttt{iostream} String Stream APIs~\cite{langer2000standard}
as the code changes required are minimal.
The key changes entail passing buffer pointers during the opening and closing of
the stream, and adding control logic to enable stream use.

In a component using the POSIX APIs, a data stream buffer is opened with either
the \texttt{fmemopen} or \texttt{open\_memstream} functions from \texttt{stdio.h}.
\texttt{open\_memstream} only supports write operations and automatically
grows as needed.
\texttt{fmemopen} supports reading and writing, but uses a fixed size, user
specified, buffer.
Once the buffer is created file read and write operations are performed through
POSIX APIs accepting the \texttt{FILE} pointer returned by the buffer opening
functions; i.e., \texttt{fread}, \texttt{fwrite}, \texttt{fscanf},
\texttt{fprintf}, etc.
After all read or write operations are complete and \texttt{fclose} is called
the buffer created with \texttt{fmemopen} is automatically deallocated.
An \texttt{open\_memstream} created buffer requires the user to deallocate it.

Example uses of the POSIX C and C++ \texttt{iostream} APIs are located in the
Appendix.
