\section{Component Interactions} \label{sec:imp_Coupling}
The design and implementation of procedures within existing software packages
directly affects how they interact with other workflow procedures.
Procedures provided by a given software package are often grouped by
functionality into a component; a reusable unit of composition with a set of
interfaces to query and modify encapsulated state
information~\cite{Szyperski02}.
The most interoperable, reusable, and extensible components are those with APIs,
minimal dependencies, minimal exposure of symbols (e.g., through use of the
unnamed namespace in C++ or the \texttt{static} prefix in C), and
scoped interfaces (e.g., via C++ \texttt{namespaces} or function name prefixes),
and no resource leaks~\cite{Miller2004,brownKnepleySmith,Gropp_1999}.
Conversely, many legacy components (e.g., analysis codes) may simply have file or
command line interfaces (i.e., they do not provide libraries with APIs) and
have little control of the symbols and memory they use.
The xSDK project formalizes these levels of interoperability and, from that,
defines basic requirements of packages for inclusion in its
ecosystem~\cite{xsdkPackagePolicies2016}.
In Sections~\ref{sec:imp_phasta} through~\ref{sec:imp_omega3p} we discuss the
design of three different analysis components and the impact of each design on
coupling with an unstructured mesh adaptation component.

In-memory component interactions are supported by bulk or atomic information
transfers.
A bulk transfer provides a large set of information following some provided
format.
For unstructured meshes this transfer could be an array of node-to-element
connectivities passed from a mesh adaptation or generation procedure to an
analysis code.
Conversely, an atomic transfer provides a single, or highly localized, piece of
information.
Continuing the connectivity example, an atomic transfer would be the nodes
bounding a single element.
In Section~\ref{sec:imp_omega3p} we provide another atomic example that computes
Jacobians of mesh elements classified on curved geometric model entities.

Our approach for high performing and scalable component interactions avoids
filesystem I/O by implementing bulk and atomic transfers with component APIs or
data streams.
Thus, component interactions in this work are within a single executable typically
built from multiple libraries.
Alternatively, the ADIOS tools provide a mechanism for the in-memory
coupling of multiple executables~\cite{bennett2012combining,zhang2012enabling}.
Likewise, Rasquin et al.~\cite{RasquinSC11Poster} demonstrated in-situ
visualization with PHASTA (see Section~\ref{sec:imp_phasta}) and
ParaView using GLEAN~\cite{glean2011}.

The type of interaction chosen to couple a pair of components depends on their
implementations.
Components with APIs that encapsulate creation, deletion, and access to
underlying data structures support in-memory interactions at different levels of
granularity.
At the finest level a developer may implement all atomic mesh entity query
functions such that components can share the same mesh structure; trading
increased development costs for lower memory usage.
An excellent example of mesh data sharing is the use of octree structures in the
development of parallel adaptive analyses~\cite{BursteddeWilcoxGhattas11}.
At a coarser level, a developer may simply create another mesh representation (a
bulk transfer) through use of interfaces encapsulating mesh construction;
trading higher memory usage for lower development costs.
Although this method will allow for in-memory integration, it can suffer from
the same disadvantages as the former approach in that a significant amount of
time and effort will be required for code modification and verification.
A generalization of this coarser level approach defines common sets of
interfaces through which all components interact.
For example, in the rotorcraft aerodynamics community the HELIOS
platform provides a set of analysis, meshing, adaptation, and load balancing
components via the Python-based Software Integration
Framework~\cite{sankaran2010application}.

Components that support a common file format and use one file
per process (e.g., POSIX C {\tt stdio.h}~\cite{posixstandard} or
C++ {\tt iostream}) can use our data stream approach with minimal software
changes.
Here, the bulk transfer is taken to nearly its highest level; the exchange of
process-level data sets.
This approach is also a logical choice for legacy analysis codes that do not
provide APIs to access or create their input and output data structures.

Using a serialization framework like Google's FlatBuffers~\cite{flatbuffers}, or
Cap'n Proto~\cite{varda2015cap}, also supports bulk data exchanges 
through use of their APIs and data layout specification mechanism.
Furthermore, some of these frameworks provide a `zero copy' mode that avoids
encode and decode overheads; the serialized information can be directly accessed
after transfer.
Like the HELIOS approach, this approach is an interesting option for components
that will be integrated with many others.

Details for implementing the bulk and atomic transfers are given in the
following sub-sections.
\subsection{Component Interfaces}\label{sec:imp_Interfaces}
The components in adaptive simulations that provide geometric model, mesh, and
field information~\cite{BeaWal,ibanez2016pumi,Ollivier10}, and the relations
between them, are essential to error estimation,
adaptation~\cite{park2008parallel,loseille2015parallel,kirk2006libmesh}, and
load balancing~\cite{SmithParma2015} services.
For example, transferring field tensors during mesh adaptation requires the
field-to-mesh relation~\cite{farrell2009conservative}.
Likewise, the mesh-to-model relation, typically called classification, and
geometric model shape information enable mesh modifications (e.g., vertex
re-positioning) that are consistent with the actual geometric
domain~\cite{BeaWal}.
The other common interactions of the mesh with the geometric domain
is in the transformation of the input field tensors onto the
mesh to define the boundary conditions, material parameters and initial
conditions~\cite{OBaBea}.
Because of this strong dependency, we provide these components and services
together as the open-source Parallel Unstructured Mesh Infrastructure
(PUMI)~\cite{ibanez2016pumi,pumi_github}.

A good example of advanced component interface usage is the PUMI
Superconvergent Patch Recovery (SPR) error estimation component.
The SPR routines estimate solution error by constructing an improved
finite element solution using a patch-level Zienkiewicz-Zhu~\cite{zz2} least
squares data fit.
SPR provides two methods which use the patch-recovery routines.
The first method recovers discontinuous solution gradients over a patch of
elements and approximates an improved solution by fitting a continuous solution
over the elemental patch.
The second method provides an improved solution in much the same way as the
first, but operates directly on integration point information obtained by the
finite element analysis.
For both methods, the improved and primal solutions are then used to create a
mesh size field that is passed to MeshAdapt to
guide mesh modification operations~\cite{alauzet2006parallel,frey2005}.

\subsection{Data Streams} \label{sec:imp_DataStreams}
Components can pass information and avoid expensive filesystem operations
through the use of buffer-based data streams.
This approach is best suited for components already using
POSIX C \texttt{stdio.h}~\cite{posixstandard}
or C++ \texttt{iostream} String Stream APIs~\cite{langer2000standard}
as the code changes required are minimal.
The key changes entail passing buffer pointers during the opening and closing of
the stream, and adding control logic to enable stream use.

In a component using the POSIX APIs, a data stream buffer is opened with either
the \texttt{fmemopen} or \texttt{open\_memstream} functions from \texttt{stdio.h}.
\texttt{open\_memstream} only supports write operations and automatically
grows as needed.
\texttt{fmemopen} supports reading and writing, but uses a fixed size, user
specified, buffer.
Once the buffer is created file read and write operations are performed through
POSIX APIs accepting the \texttt{FILE} pointer returned by the buffer opening
functions; i.e., \texttt{fread}, \texttt{fwrite}, \texttt{fscanf},
\texttt{fprintf}, etc.
After all read or write operations are complete and \texttt{fclose} is called
the buffer created with \texttt{fmemopen} is automatically deallocated.
An \texttt{open\_memstream} created buffer requires the user to deallocate it.

Example uses of the POSIX C and C++ \texttt{iostream} APIs are located in the
Appendix.

