Albany~\cite{salinger2013albany,albany2016} is a parallel, implicit,
unstructured mesh, multi-physics, finite element code used for the solution and analysis of partial
differential equations.
The code is built on over 100 software components and heavily leverages packages
from the Trilinos project~\cite{TrilinosOverview}.
Both Albany and Trilinos adopt an `agile components' approach
to software development that emphasizes interoperability.
Albany has been used to solve a variety of multi-physics problems
including ice sheet modeling and modeling the mechanical response of nuclear
reactor fuel.
The largest Albany runs have had over a billion degrees of freedom and
used over 32Ki cores.
Albany's high performance, generality, and component-based design made it an
ideal candidate for the construction of an in-memory adaptive workflow using
bulk API-based transfers.

The Albany analysis code provides an abstract base class for the mesh
discretization.
Implementing the class with PUMI's complete topological mesh representation
simply required understanding Albany's discretization structures.
Like most finite element codes, Albany stores a list of mesh
nodes and a node-to-element connectivity map to define mesh elements.
Albany's Dirichlet and Neumann boundary conditions though, need additional data
structures.
The Dirichlet boundary condition data structure is simply an array of
constrained mesh nodes.
The more complex Neumann boundary condition structure requires lists of mesh
elements associated with constrained mesh faces; a classification check followed
by a face-to-element upward adjacency query.
Algorithm~\ref{alg:sidesets} details this process.
Here the notation $M^d_j$ ($G^d_j$) refers to the $j^{th}$ mesh (model)
entity of dimension $d$,
$M^d_j \sqsubset G$ returns the geometric model classification of $M^d_j$, and
$\{M^d_i\{M^q\}\}$ is the set of mesh entities of dimension $q$ that are
adjacent to $M^d_i$.
The PUMI implementation of Albany's discretization and boundary
condition structures allows us to define and solve complex problems without
having to create a second complex mesh data structure (e.g., a Trilinos STK
mesh).

\begin{algorithm}
  \caption{Construction of Neumann Boundary Condition Structure}
  \label{alg:sidesets}
  \begin{algorithmic}[1]
    \State // store the mapping of geometric model faces to \texttt{side\_sets}
    \State \texttt{invMap} $\gets$ mapping of $G^2_j$ to \texttt{side\_sets}
    \ForAll{ $M^2_i$ $\in$ $\{M^2\}$ }
        \State // get the geometric model classification of the mesh face
        \State $G^d_j$ $\gets$ $M^2_i \sqsubset G$
        \If{ $G^d_j$ $\notin$ \texttt{invMap}}
          \State // not a geometric model face associated with a \texttt{side\_set}
          \State continue
        \EndIf
        \State // for simplicity of the example we assume the model is manifold
        \State // upward adjacent element to the mesh face
        \State $M^3_j$ $\gets$ $\{M^2_i\{M^3\}\}$
        \State // collect additional element and face info
        \State \texttt{elm\_LID} $\gets$ local id of $M^3_j$
        \State \texttt{side\_id} $\gets$ local face index of $M^2_i$
        \State \texttt{side\_struct} $\gets$
               \{\texttt{elm\_LID},\texttt{side\_id},$M^3_j$\}
        \State insert \texttt{side\_struct} into \texttt{side\_set\_list}
      \EndFor
  \end{algorithmic}
\end{algorithm}
