\subsection{Component Interfaces}\label{sec:imp_Interfaces}
The components in adaptive simulations that provide geometric model, mesh, and
field information~\cite{BeaWal,ibanez2016pumi,Ollivier10}, and the relations
between them, are essential to error estimation,
adaptation~\cite{park2008parallel,loseille2015parallel,kirk2006libmesh}, and
load balancing~\cite{SmithParma2015} services.
For example, transferring field tensors during mesh adaptation requires the
field-to-mesh relation~\cite{farrell2009conservative}.
Likewise, the mesh-to-model relation, typically called classification, and
geometric model shape information enable mesh modifications (e.g., vertex
re-positioning) that are consistent with the actual geometric
domain~\cite{BeaWal}.
The other common interactions of the mesh with the geometric domain
is in the transformation of the input field tensors onto the
mesh to define the boundary conditions, material parameters and initial
conditions~\cite{OBaBea}.
Because of this strong dependency, we provide these components and services
together as the open-source Parallel Unstructured Mesh Infrastructure
(PUMI)~\cite{ibanez2016pumi,pumi_github}.

A good example of advanced component interface usage is the PUMI
Superconvergent Patch Recovery (SPR) error estimation component.
The SPR routines estimate solution error by constructing an improved
finite element solution using a patch-level Zienkiewicz-Zhu~\cite{zz2} least
squares data fit.
SPR provides two methods which use the patch-recovery routines.
The first method recovers discontinuous solution gradients over a patch of
elements and approximates an improved solution by fitting a continuous solution
over the elemental patch.
The second method provides an improved solution in much the same way as the
first, but operates directly on integration point information obtained by the
finite element analysis.
For both methods, the improved and primal solutions are then used to create a
mesh size field that is passed to MeshAdapt to
guide mesh modification operations~\cite{alauzet2006parallel,frey2005}.
