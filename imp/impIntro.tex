\section{Introduction}

Simulations on massively parallel systems are most effective when data
movement is minimized.
Data movement costs increase with the depth of the memory hierarchy; a design
trade-off for increased capacity.
For example, the lowest level on-node storage in the IBM Blue Gene/Q A2
processor~\cite{haring2012ibm} is the per core 16KiB L1 cache (excluding
registers) and has a peak bandwidth of 819 GiB/s.
The highest level on-node storage, 16GiB of DDR3 main memory, provides a million
times more capacity but at a greatly reduced bandwidth of 43GiB/s,
1/19$th$ of L1 cache~\cite{lo2014roofline}.
One level further up the hierarchy is the parallel filesystem\footnote{For the
sake of simplicity we assume that the main memory of other nodes is not
available.
But, there are checkpoint-restart methods that use local and remote memory for
increased performance
~\cite{rma-fault-tolerance-2014,compression-cr-2012}.
}.
At this level, the bandwidth and capacity relationship are again less favorable
and further compromised by the fact that the filesystem is a shared resource.
Table~\ref{tbl:systems} lists the per node peak main memory and filesystem
bandwidth across five generations of Argonne National Laboratory leadership
class systems: Blue Gene/L~\cite{yu2006high,adiga2002overview}, Intrepid
Blue Gene/P~\cite{lang2009performance,alam2008early}, Mira
Blue Gene/Q~\cite{haring2012ibm,bui2014scalable},
Theta~\cite{knl,jeffers2016intel}, and 2018's Aurora~\cite{aurorafacts}.
Based on these peak values the bandwidth gap between main memory and the
filesystem is at least three orders of magnitude.
Software must leverage the cache and main memory bandwidth performance advantage
during as many workflow operations as possible to maximize performance.

\begin{table}[h] \centering
  \caption[
  Per node main memory and filesystem peak bandwidth over five
  generations of Argonne National Laboratory systems.
  ]{Per node main memory and filesystem peak bandwidth over five
  generations of Argonne National Laboratory systems.
  The values in parentheses indicate the increase relative to
  the previous generation system.}
\label{tbl:systems}
\begin{tabular}{l|cc}
        & Memory BW  & Filesystem BW \\
        & (GiB/s)    & (GiB/s)    \\
 \hline
 BG/L   & 5.6        & 0.0039         \\
 BG/P   & 14 (2.4x)  & 0.0014 (0.36x)   \\
 BG/Q   & 43 (3.1x)  & 0.0049 (3.5x) \\
 Theta  & 450 (11x)  & 0.058 (12x) \\
 Aurora & 600 (1.3x) & 0.020 (0.34x)
\end{tabular}
\end{table}
