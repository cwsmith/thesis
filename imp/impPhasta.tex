PHASTA solves complex fluid flow
problems~\cite{Gal10,Rod13,sahni2009scalable,Zhou-flow,phasta_github}
on up to 768Ki cores with 3Mi ($3*2^{20}$) MPI processes~\cite{rasquinCise2014} using a
stabilized finite element method~\cite{WhiJan01} primarily implemented with
FORTRAN77 and FORTRAN90.
Support for mesh adaptivity, dynamic load balancing, and reordering has
previously been provided by the C++ PUMI-based component,
chef, through file I/O.
This file-based coupling uses a format and procedures that were originally
developed over a decade ago.
Our work adds support for PHASTA and chef in-memory bulk data stream transfers.
We show performance of this approach with a multi-cycle test using a fixed
size mesh and present an adaptive, two-phase dam-break analysis.

The data stream approach for in-memory interactions was the logical choice
given the existing POSIX file support, and the lack of PHASTA
interfaces to modify FORTRAN data structures.
The chef and PHASTA data stream implementation maintains
support for POSIX file-based I/O by replacing direct calls to POSIX file open,
read and write routines with function pointers.

Our work also adds a few execution control APIs to run PHASTA within
an adaptive workflow.
The API implementation uses the singleton design
pattern~\cite{andreiDesignPatterns} and several of Miller's Smart Library
techniques~\cite{Miller2004}.
This approach provides backward compatibility for legacy execution modes,
such as scripted file-based adaptive loops, with minimal code changes and easily
accounts for the heavy reliance on global data common to legacy FORTRAN codes.
